\documentclass[11pt]{article}

\usepackage{amsmath}
\usepackage{mathtools}
\usepackage{geometry}
\usepackage[parfill]{parskip}
\usepackage{graphicx}
\usepackage{amssymb}
\usepackage{epstopdf}

\geometry{a4paper}

\begin{document}

\title{Load Flow Notes}
\author{Dan Gordon}
\date{Today}
\maketitle

\section{The model}
\subsection{Busses and lines}
$N+1$ busses, with bus 0 being slack bus.

Sign conventions: Voltages are expressed relative to ground. Positive power at a bus = power consumed (e.g. flowing to ground).

\begin{description}
\item[Slack]$V$ is constant, $V_0$.
\item[PQ]Complex power $S$ is constant. $S_i = S^0_i$.
\item[PV]Real power $P$ is constant, $P_i = P_{0,i}$. Voltage magnitude $|V|$ is constant, $|V_i| = V_{0,i}$.
\end{description}

\begin{flalign}
	y_{ik} &= \text{admittance between nodes $i$,  $k$.} &\\
	I_{0,i} &= \text{constant current injection into node $i$} &\\
	y_i &= \text{shunt admittance at node $i$}&
\end{flalign}

\subsection{Y-matrix}
\begin{align}
	Y_{ik} = \begin{cases}
		-y_{ik}&\text{if $i \ne k$} \\
		y_i + \sum_l y_{il}& \text{if $i = k$}
	\end{cases}
	\label{EQ_Y_MATRIX}
\end{align}

\subsection{Power balance equations}
\label{SEC_PBEQS}
\begin{align}
S^*_i = V^*_i\left[-I_{0,i}+ \sum_{k=0}^N{Y_{ik}V_k}\right]
\label{EQ_PB_CPLEX}
\end{align}
Or, using the real and imaginary power
\begin{align}
	P_i &= M_i\sum_kM_k\left[G_{ik}\cos(\theta_i-\theta_k)+B_{ik}\sin(\theta_i - \theta_k)\right]  \label{EQ_PB_REAL} \\
	Q_i &= M_i\sum_kM_k\left[G_{ik}\sin(\theta_i-\theta_k)-B_{ik}\cos(\theta_i - \theta_k)\right] \label{EQ_PB_IMAG}
\end{align}
where $V = M\exp(j\theta)$, $S = P + jQ$, $Y = G + jB$.

For PQ busses, $M$ and $\theta$ are the unknowns. Each PQ bus therefore has two unknowns and two equations associated with it. For PV busses, $Q$ and $\theta$ are unknowns. However, if we remove the equation for $Q$ for each PV bus, then that bus will contribute one equation and one unknown to the system, and the system will still be well determined. After it is solved, $Q$ can immediately be recovered using Eq. \ref{EQ_PB_IMAG}.

\section{Newton-Raphson}
Ignoring for now constant current injections, and using Eqs. \ref{EQ_PB_REAL} and \ref{EQ_PB_IMAG}, we have
\begin{align}
	g_i &= -\frac{P_i}{M_i} + \sum_kM_k\left[G_{ik}\cos(\theta_i-\theta_k)+B_{ik}\sin(\theta_i - \theta_k)\right]= 0 \notag \\
	h_i &= -\frac{Q_i}{M_i} + \sum_kM_k\left[G_{ik}\sin(\theta_i-\theta_k)-B_{ik}\cos(\theta_i - \theta_k)\right]= 0
\end{align}
where the second equation is only used for PQ busses, see the discussion in Section \ref{SEC_PBEQS}.

Define  $x = [M_{\text{\tiny PQ}}, \theta_{\text{\tiny PQ}},\theta_{\text{\tiny PV}}]^T$. Define $f(x) = [g_{\text{\tiny PQ}}(x), g_{\text{\tiny PV}}(x), h_{\text{\tiny PQ}}(x)]^T$.

The Jacobian $\partial f_\alpha/\partial x_\beta$ is therefore given by the following expressions:
\begin{align}
\frac{\partial g_i}{\partial M_l} &= \frac{P_i}{M_l^2}\delta_{il}+G_{il}\cos(\theta_i - \theta_l) + B_{il}\sin(\theta_i - \theta_l)  \notag \\
\frac{\partial g_i}{\partial \theta_l} &= M_l\left\{G_{il}\sin(\theta_i-\theta_l)-B_{il}\cos(\theta_i-\theta_l)\right\} \notag \\
	&+ \delta_{il}\sum_kM_k\left\{-G_{ik}\sin(\theta_i-\theta_k)+B_{ik}\cos(\theta_i-\theta_k)\right\} \notag \\
\frac{\partial h_i}{\partial M_l} &= \frac{Q_i}{M_i^2}\delta_{il}+G_{il}\sin(\theta_i - \theta_l) - B_{il}\cos(\theta_i - \theta_l)  \notag \\
\frac{\partial h_i}{\partial \theta_l} &= M_l\left\{-G_{il}\cos(\theta_i-\theta_l)-B_{il}\sin(\theta_i-\theta_l)\right\} \notag \\
	&+ \delta_{il}\sum_kM_k\left\{G_{ik}\cos(\theta_i-\theta_k)+B_{ik}\sin(\theta_i-\theta_k)\right\} \notag \\
\end{align}

The NR method puts
\begin{align}
x_{n+1} = x_n - [J(x_n)]^{-1}f(x_n)
\end{align}

The slow part is computing the Jacobian at each stage.

\subsection{Holomorphic embedding}
Embed in equations that depend on variable $s$, st. equations are holomorphic. At $s$ = 0, get equations of isolated busses plus slack bus. At $s = 1$, recover original equations.
For PQ busses,
\begin{align}
	sS^*_i = V^*_i(s*)\left[
		-sI_{0i} + sy_iV_i(s) + Y'_{i0}V_0 + \sum_{k=1}^N{Y'_{ik}V_k(s)}
	\right]
	\label{EQ_POWER_BAL}
\end{align}
where $Y'$ is $Y$ minus the shunt terms. 
For PV busses, we only want the real power balance equation:
\begin{align}
	sP_i = \frac{1}{2}\left\{
		\begin{aligned}
			&V^*_i(s^*)\left[
				-sI_{0i} + sy_iV_i(s) + Y'_{i0}V_0 + \sum_{k=1}^N{Y'_{ik}V_k(s)}
			\right] \\
			+ &V_i(s)\left[
				-sI^*_{0i} + sy^*_iV^*_i(s^*) + Y'^*_{i0}V^*_0 + \sum_{k=1}^N{Y'^*_{ik}V^*_k(s^*)}
			\right]
		\end{aligned}
	\right\}
	\label{EQ_POWER_BAL}
\end{align}

Now, for real $c$, $d$, we create the following series expansion:
\begin{align}
	V_i(s) &= V_0 + \sum_{n=1}^{\infty}{s^n (c_{i,n} + j d_{i,n})},
\end{align}

For PV busses, we require
\begin{align}
	V_0^2 + s(M_i^2 - V_0^2) = \left[V_0 + \sum_{n=1}^{\infty}{s^n (c_{i,n} + j d_{i,n})}\right]\left[V_0 + \sum_{m=1}^{\infty}{s^m (c_{i,m} - j d_{i,m})}\right] 
\end{align}
which implies,
\begin{align}
	c_{i,n} =  \frac{M_i^2 - V_0^2}{2V_0}\delta_{n1} - \frac{1}{2V_0}\sum_{m=1}^{n-1}\left(c_{i,m}c_{i,n-m}+d_{i,m}d_{i,n-m}\right)
	\label{EQ_B}
\end{align}
Thus, for PV busses, if we know all $c_{i,m}$ and $d_{i,m}$ for $m < n$ , we can immediately find $c_{i, n}$.
\subsection{Bus equations}
The busses then satisfy the following equations:
\begin{align}
\begin{split}
	sS^*_i &= \left[V_0 + \sum_{n=1}^{\infty}s^n (c_{i,n} - j d_{i,n})\right] \\
	&\times \left[
		V_0Y'_{i0} + \sum_{k=1}^N Y'_{i,k}V_0 -sI_{0i}+ \sum_{m = 1}^{\infty}s^m \left\{
			\begin{aligned}
				&
				y_i (c_{i,m-1} + jd_{i,m-1}) \\
				&+ \sum_{k=1}^NY'_{ik}(c_{k,m} + jd_{k,m})
			\end{aligned}
		\right\}
	\right]
\end{split}
\end{align}
and, using the fact that
\begin{align}
	\sum_{k=0}^N{Y'_{ik}} = 0,
\end{align}
we have
\begin{align}
\begin{split}
	sS^*_i &= \left[V_0 + \sum_{n=1}^{\infty}s^n (c_{i,n} - j d_{i,n})\right] \\
	&\times \left[
		-sI_{0i} + \sum_{m = 1}^{\infty}s^m \left[
			y_i (c_{i,m-1} + jd_{i,m-1}) + \sum_{k=1}^NY'_{ik}(c_{k,m} + jd_{k,m})
		\right]
		\right]
\end{split}
\label{EQ_MISC1}
\end{align}
We can equate powers of $s$ separately, thus,
\begin{align}
	S^*_i\delta_{n1} &=
			V_0\left[-I_{0i}\delta_{n1} + y_i(c_{i,n-1} + j d_{i,n-1})+ \sum_{k=1}^NY'_{ik}(c_{k,n} + jd_{k,n})\right] \notag \\
			&-I_{0i}\left(c_{i,n-1}-jd_{i,n-1}\right) \notag \\
			&+ \sum_{m=1}^{n-1}(c_{i,n-m}-jd_{i,n-m})
			\left[
				y_i(c_{i,m-1} + jd_{i,m-1})+ \sum_{k=1}^NY'_{ik}(c_{k,m} + jd_{k,m}) 
			\right]
\end{align}
or, taking real and imaginary parts:
\begin{align}
	P_i\delta_{n1} &=
		V_0\left[-\Re(I_{0i})\delta_{n1} + g_ic_{i,n-1} - b_i d_{i,n-1}+ \sum_{k=1}^N\left(G'_{ik}c_{k,n} - B'_{ik}d_{k,n}\right)\right] \notag \\
		&- \Re(I_{0i})c_{i,n-1} - \Im(I_{0i})d_{i, n-1} \notag \\
		&+ \sum_{m=1}^{n-1}{
			c_{i,n-m}
			\left[
				g_ic_{i,m-1} - b_id_{i,m-1}+ \sum_{k=1}^N\left(G'_{ik}c_{k,m} - B'_{ik}d_{k,m}\right) 
			\right]
		} \notag \\
		&+ \sum_{m=1}^{n-1}{
			d_{i,n-m}
			\left[
				g_id_{i,m-1} + b_ic_{i,m-1}+ \sum_{k=1}^N\left(G'_{ik}d_{k,m} + B'_{ik}c_{k,m}\right) 
			\right]
		}
		\label{EQ_HELM_R_1_A} \\
	-Q_i\delta_{n1} &=
			V_0\left[-\Im(I_{0i})\delta_{n1} + g_id_{i,n-1} + b_i c_{i,n-1} + \sum_{k=1}^N\left(G'_{ik}d_{k,n} + B'_{ik}c_{k,n}\right)\right] \notag \\
			&- \Im(I_{0i})c_{i, n-1} + \Re(I_{0i})d_{i, n-1} \notag \\
			&+ \sum_{m=1}^{n-1}{
				c_{i,n-m}
				\left[
					g_id_{i,m-1} + b_ic_{i,m-1} + \sum_{k=1}^N\left(G'_{ik}d_{k,m} + B'_{ik}c_{k,m}\right) 
				\right]
		} \notag \\
		&- \sum_{m=1}^{n-1}{
			d_{i,n-m}
			\left[
				g_ic_{i,m-1} - b_id_{i,m-1}+ \sum_{k=1}^N\left(G'_{ik}c_{k,m} - B'_{ik}d_{k,m}\right) 
			\right]
		} 
		\label{EQ_HELM_IM_1_A}
\end{align}

The PV busses must satisfy only the real equation, Eq. (\ref{EQ_HELM_R_1}) - see various discussions above. For PV busses, $c_n$ is not considered an unknown (see above), so there are $2N_{\text{\tiny PQ}} + N_{\text{\tiny PV}}$ real equations and the same number of unknowns.

We rewrite Eqs. (\ref{EQ_HELM_R_1_A}) and (\ref{EQ_HELM_IM_1_A}), putting the unknowns on the left:
\begin{align}
	 &\sum_{k\in\text{\tiny PQ}}(-V_0G'_{ik}c_{k,n}) + \sum_{k\in\text{\tiny PQ,PV}}V_0B'_{ik}d_{k,n} =\sum_{k\in\text{\tiny PV}} V_0G'_{ik}c_{k,n} \notag \\
		&\hspace{1cm}+ \delta_{n1} \left[-P_i - V_0\Re(I_{0i})\right] + c_{i,n-1}\left[V_0g_i - \Re(I_{0,i})\right]+d_{i,n-1}\left[- V_0b_i  - \Im(I_{0,i})\right] \notag \\
		&\hspace{1cm}+ \sum_{m=1}^{n-1}{
			c_{i,n-m}
			\left[
				g_ic_{i,m-1} - b_id_{i,m-1}+ \sum_{k=1}^N\left(G'_{ik}c_{k,m} - B'_{ik}d_{k,m}\right) 
			\right]
		} \notag \\
		&\hspace{1cm}+ \sum_{m=1}^{n-1}{
			d_{i,n-m}
			\left[
				b_ic_{i,m-1} + g_id_{i,m-1} + \sum_{k=1}^N\left(B'_{ik}c_{k,m} + G'_{ik}d_{k,m}\right) 
			\right]
		}
		\label{EQ_HELM_R_1} \\
	 &\sum_{k\in\text{\tiny PQ}}(-V_0B'_{ik}c_{k,n}) + \sum_{k\in\text{\tiny PQ,PV}}(-V_0G'_{ik}d_{k,n}) = \sum_{k\in\text{\tiny PV}}V_0B'_{ik}c_{k,n} \notag \\
		&\hspace{1cm}+\delta_{n1}\left[Q_i - V_0\Im(I_{0i})\right] + c_{i,n-1}\left[V_0 b_i-\Im(I_{0i})\right] + d_{i,n-1}\left[V_0g_i  + \Re(I_{0i})\right]\notag \\
		&\hspace{1cm}+ \sum_{m=1}^{n-1}{
			c_{i,n-m} \left[
				b_ic_{i,m-1} + g_id_{i,m-1}  + \sum_{k=1}^N\left(B'_{ik}c_{k,m} + G'_{ik}d_{k,m}\right) 
			\right],
		} \notag \\
		&\hspace{1cm}+ \sum_{m=1}^{n-1}{
			d_{i,n-m} \left[
				-g_ic_{i,m-1} + b_id_{i,m-1}+ \sum_{k=1}^N\left(-G'_{ik}c_{k,m} + B'_{ik}d_{k,m}\right) 
			\right].
		} 
		\label{EQ_HELM_IM_1}
\end{align}
The second equation applies only for PQ busses.

\section{Solution method}
\begin{enumerate}
	\item 
		Write down the following:
		\begin{enumerate}
			\item
				$V_0$, the (real) slack voltage.
			\item
				For each PQ bus, the complex power $S_i$,
			\item
				For each PV bus, the real power $P_i$ and the voltage magnitude squared, $V_i^2$,
			\item
				$I_{0i}$, the constant current injection.
			\item
				The shunt admittances, $y_i$,
			\item
				The $Y'$ matrix,
				\begin{align}
					Y'_{ik} = \begin{cases}
						-y_{ik}&\text{if $i \ne k$} \\
						\sum_l y_{il}& \text{if $i = k$}
					\end{cases}
					\label{EQ_YDASH_MATRIX}
				\end{align}
		\end{enumerate}
	\item 
		Iterate over the series expansion order $n$. It is assumed we know all variables at $n-1$.
		\begin{enumerate}
			\item
				For PV busses, calculate $b_{i,n}$ using Eq. \ref{EQ_B}. For PQ busses, we know that $a_{i,n}$ is 0.
			\item
				Solve the equations \ref{EQ_PB_PQ} and \ref{EQ_PB_PV} to find the unknowns.
			\item
				Iterate.
		\end{enumerate}
\end{enumerate}
\section{Simplification: One slack bus, one PQ bus, no current injections or shunts, all real quantities}
An analytic solution can easily be calculated as
\begin{align}
U = \frac{1}{2} \pm \frac{1}{2}\sqrt{1 + 4\sigma}
\end{align}
where $U = V/V_0$ and $\sigma = P/(GV_0^2)$.
\section{Simplification: One slack bus, one PV bus, no current injections or shunts}
An analytic solution can be calculated:
\begin{align}
	U &= V/V_0 \notag \\
	K &= \frac{P}{BV_0^2}-\frac{G|U^2|}{B}
\end{align}
Now solve for $U_R$
\begin{align}
	U_R^2 + U_R\frac{2KGB}{G^2 + B^2}+\frac{B^2(K^2-|U|^2}{G^2+B^2}
\end{align}
And obtain $U_I$
\begin{align}
	U_I &= \pm\sqrt{|U|^2-U_R^2} 
\end{align}
And finally
\begin{align}
	Q = V_0^2\left[GU_I-B(|U|^2-U_R)\right]
\end{align}
\section{Simplification: One slack bus, one PV bus, no current injections or shunts, all real quantities}
An analytic solution can easily be calculated as
\begin{align}
	V_R &= \frac{\left|V\right|^2}{V_0}-\frac{P}{GV_0} ,\notag \\
	V_I &= \pm\sqrt{\left|V\right|^2 - V_R^2}, \notag \\
	Q &= -GV_0V_I.
\end{align}
\section{More efficient NR}
If we rewrite the power flow equations, we can obtain a Y matrix that has large sections linear in $V$. This means that large parts of the Jacobian will not depend on $V$, and hence don't need to be recalculated at each pass.

We have (ignoring for now the load model)
\begin{align}
I_i = \frac{S_i^*}{V_i^*} = \sum_{k=0}^NY_{ik}V_k
\end{align}
Define the current mismatch:
\begin{align}
\Delta_i = -\frac{S^*_i}{V_i^*} + \sum_{k=0}^NY_{ik}V_k
\end{align}

Note that we can treat $V$ and $V^*$ as independent variables in the following calculation of the Jacobian. 
\footnote{
	For example, for a function $f(V, V^*)$ we have
	\begin{align}
	df &= \frac{\partial f}{\partial V} \frac{\partial V}{\partial V_r}dV_r  +  \frac{\partial f}{\partial V} \frac{\partial V}{\partial V_i}dV_i +  \frac{\partial f}{\partial V^*} \frac{\partial V^*}{\partial V_r}dV_r  + \frac{\partial f}{\partial V^*} \frac{\partial V^*}{\partial V_i}dV_i \notag \\
	&= \frac{\partial f}{\partial V} dV_r  + j\frac{\partial f}{\partial V}dV_i +  \frac{\partial f}{\partial V^*}dV_r  - j\frac{\partial f}{\partial V^*} dV_i \notag \\
	&= \frac{\partial f}{\partial V} dV +  \frac{\partial f}{\partial V^*}dV^* \notag
	\end{align}
}
\begin{align}
\frac{\partial \Delta_i}{\partial V_k} &= Y_{ik} + \frac{\partial S^*_i}{\partial V_k}, &\frac{\partial \Delta_i}{\partial V^*_k} &=  \frac{S^*_i}{{V^*_i}^2}\delta_{ik} + \frac{\partial S^*_i}{\partial V^*_k},  \notag \\
\frac{\partial \Delta^*_i}{\partial V_k} &= \frac{S_i}{{V_i}^2}\delta_{ik} + \frac{\partial S_i}{\partial V_k}, &\frac{\partial \Delta^*_i}{\partial V^*_k} &= Y^*_{ik} + \frac{\partial S_i}{\partial V^*_k},
\end{align}

However, the treatment above gives a $2N \times 2N$ matrix. At the expense of brevity, it may be better numerically to work with the real and imaginary parts of $V$. Also, numerical experiments show faster convergence, probably because the configuration space has half as many dimensions.
\begin{align}
\Delta_i &= -\frac{S^*_i\left(V_{Ri}+jV_{Ii}\right)}{M^2} + \sum_{k=0}^NY_{ik}\left(V_{Rk} + j V_{Ik}\right)
\end{align}
or,
\begin{align}
\Delta_{Ri} &= \frac{-P_iV_{Ri}-Q_iV_{Ii}}{M^2_i} + \sum_{k=0}^N\left(G_{ik}V_{Rk} - B_{ik}V_{Ik}\right)\notag \\
\Delta_{Ii} &= \frac{-P_iV_{Ii}+Q_iV_{Ri}}{M^2_i} + \sum_{k=0}^N\left(G_{ik}V_{Ik} + B_{ik}V_{Rk}\right)
\end{align}
Where $M^2_i$ is the voltage magnitude, $M^2_i := V_{Ri}^2 + V_{Ii}^2$. The Jacobian is then
\begin{align}
\frac{\partial \Delta_{Ri}}{\partial V_{Rk}} &= \left[\frac{-P_i}{M^2_i} + 2V_{Ri}\frac{P_iV_{Ri} + Q_iV_{Ii}}{M^4_i}\right]\delta_{ik}+G_{ik} 
+\frac{-V_{Ri}\partial P_i/\partial V_{Rk}-V_{Ii}\partial Q_i/\partial V_{Rk}}{M^2_i}\notag \\
\frac{\partial \Delta_{Ri}}{\partial V_{Ik}} &= \left[\frac{-Q_i}{M^2_i} + 2V_{Ii}\frac{P_iV_{Ri} + Q_iV_{Ii}}{M^4_i}\right]\delta_{ik}-B_{ik}
+\frac{-V_{Ri}\partial P_i/\partial V_{Ik}-V_{Ii}\partial Q_i/\partial V_{Ik}}{M^2_i}\notag \\
\frac{\partial \Delta_{Ii}}{\partial V_{Rk}} &= \left[\frac{Q_i}{M^2_i} + 2V_{Ri}\frac{P_iV_{Ii} - Q_iV_{Ri}}{M^4_i}\right]\delta_{ik}+B_{ik}
+\frac{-V_{Ii}\partial P_i/\partial V_{Rk}+V_{Ri}\partial Q_i/\partial V_{Rk}}{M^2_i} \notag \\
\frac{\partial \Delta_{Ii}}{\partial V_{Ik}} &= \left[\frac{-P_i}{M^2_i} + 2V_{Ii}\frac{P_iV_{Ii} - Q_iV_{Ri}}{M^4_i}\right]\delta_{ik}+G_{ik}
+\frac{-V_{Ii}\partial P_i/\partial V_{Ik}+V_{Ri}\partial Q_i/\partial V_{Ik}}{M^2_i}
\end{align}

\subsection{PQ Busses}
A generalised PQ bus may have a constant power element, a constant current element or a constant shunt admittance element. Thus
\begin{align}
S_i &= S_{Ci} + V_iI^*_{Ci} - V_iV^*_iy^*_{Ci}.
\end{align}
(The negative sign on the last term is because we are working with power and current injections into the bus: a positive voltage will give a negative injection for a shunt admittance. Also beware that $I$ is expressed as an injection, so a constant current load would have a negative $I$.)
In this case, we have
\begin{align}
\Delta_{Ri} &= \frac{-P_{Ci}V_{Ri}-Q_{Ci}V_{Ii}}{M^2_i} - I_{Ri} + V_{Ri}y_{Ri} - V_{Ii}y_{Ii} + \sum_{k=0}^N\left(G_{ik}V_{Rk} - B_{ik}V_{Ik}\right)\notag \\
\Delta_{Ii} &= \frac{-P_{Ci}V_{Ii}+Q_{Ci}V_{Ri}}{M^2_i} - I_{Ii} -V_{Ri}I_{Ii} - V_{Ii}y_{Ri} + \sum_{k=0}^N\left(G_{ik}V_{Ik} + B_{ik}V_{Rk}\right)
\end{align}
The Jacobian is now
\begin{align}
\frac{\partial \Delta_{Ri}}{\partial V_{Rk}} &= \left[\frac{-P_{Ci}}{M^2_i} + 2V_{Ri}\frac{P_{Ci}V_{Ri} + Q_{Ci}V_{Ii}}{M^4_i} + y_{Ri}\right]\delta_{ik}+G_{ik} \notag \\
\frac{\partial \Delta_{Ri}}{\partial V_{Ik}} &= \left[\frac{-Q_{Ci}}{M^2_i} + 2V_{Ii}\frac{P_{Ci}V_{Ri} + Q_{Ci}V_{Ii}}{M^4_i} - y_{Ii}\right]\delta_{ik}-B_{ik} \notag \\
\frac{\partial \Delta_{Ii}}{\partial V_{Rk}} &= \left[\frac{Q_{Ci}}{M^2_i} + 2V_{Ri}\frac{P_{Ci}V_{Ii} - Q_{Ci}V_{Ri}}{M^4_i} + y_{Ii}\right]\delta_{ik}+B_{ik} \notag \\
\frac{\partial \Delta_{Ii}}{\partial V_{Ik}} &= \left[\frac{-P_{Ci}}{M^2_i} + 2V_{Ii}\frac{P_{Ci}V_{Ii} - Q_{Ci}V_{Ri}}{M^4_i} + y_{Ri}\right]\delta_{ik}+G_{ik}
\end{align}
Note: in this formulation, the shunts are not included in $G$ and $B$ because we are explicitly accounting for them as part of the load. Note that this expression splits into a $V$ dependent part that needs to be recalculated at each iteration and a constant part that doesn't depend on $V$. In practice, we can absorb the shunt admittances as diagonal terms in $G$ and $B$. From then on, we can just forget about the shunt admittances in all calculations. So all we need to do for generalised PQ busses is to add an extra constant current term into $\Delta$.
\subsection{PV Busses}
For PV busses, the unknowns are 

\section{General branch admittances}
For a simple line, $I_{ik} = y_{ik}V_{ik}$ defines both the current flowing from $i$ into the line and the current flowing from the line to $k$. However, for more complex cases, like transformers (which internally can be represented as equivalent two port networks), the current in is not necessarily the current out. We interpret $I_{ik}$ to mean the current injected from the link $ik$ to the node $k$, and $I_{ki}$ to mean the current injected from the link $ki$ to the node $i$.

Then the general two port equation for link $l:12$ is
\begin{align}
	\begin{bmatrix}
		I_{12} \\
		I_{21}
	\end{bmatrix}_l
	&=
	\begin{bmatrix}
		Y_{11} & Y_{12} \\
		Y_{21} & Y_{22}
	\end{bmatrix}_l
		\begin{bmatrix}
		V_1 \\
		V_2
	\end{bmatrix}
\end{align}
For example, a simple line $l:ik$ with admittance $y$ has
\begin{align}
	Y_l &=
	\begin{bmatrix}
		y & -y \\ -y & y
	\end{bmatrix}
\end{align}
Which is familiar as the standard $Y$ matrix for a single line.

Let's formulate the power balance equations using this framework. The current injection at each node $i$ is
\begin{align}
I_i &= y_iV_i + \sum_{l:ik_l} I_{l;ik_l} \notag \\
&=  y_iV_i + \sum_{l:ik_l} Y_{l;ik_l}V_{k_l}
\end{align}
Now, let the total bus admittance matrix $Y_{ik}$ be defined as
\begin{align}
	Y_{ik} = y_i\delta_{ik} + \sum_{l:ik}Y_{l;ik}
\end{align}
Then we have the familiar equation
\begin{align}
	I_i &= \sum_k Y_{ik}V_k
\end{align}
 The point of this discussion is that it shows how to form the total bus admittance matrix out of the individual link admittance matrices.

The discussion generalises readily to a three phase situation, where each link can be represented as a 6-port. In such a situation, the shunt admittances could also mix phases. The shunt terms for bus $i$ can then be determined by considering the ground as a three phase bus with zero voltage on each phase, and considering a general link between $i$ and this ground bus.

In summary, to find the $Y$ matrix for a three phase power system, we need the $6\times6$ $Y$ matrices for each link including shunts. $Y$ is then just the sum of all of these, mapped to the correct indices.

\section{Transformers}
An ideal transformer has
\begin{align}
	\frac{V_1}{V_2} &= \frac{N_1}{N_2} \notag \\
	\frac{I_1}{I_2} &= \frac{N_2}{N_1} \notag \\
	\frac{S_1}{S_2} &= 1
\end{align}
This is treated as a line, so that 

\section{Three Phase Load Flow}
Essentially we want to reduce this to a series of one phase load flow problems. The problem is really one of modelling.

We use the terms node and link to represent single voltage aggregation points 
\end{document}